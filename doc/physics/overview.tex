\documentclass{article}

\usepackage{amsmath}
\usepackage{listings}
\usepackage{xcolor}
\usepackage{cleveref}
\usepackage{amsfonts}

\lstset{
	language=Python,
	backgroundcolor=\color{white},
	basicstyle=\ttfamily,
	keywordstyle=\color{blue},
	stringstyle=\color{red},
	commentstyle=\color{green},
	numberstyle=\tiny\color{gray},
	numbers=left,
	stepnumber=1,
	numbersep=10pt,
	showstringspaces=false,
	tabsize=4,
}

\begin{document}
	\section{Collision physics}
	\subsection{Collision handling}

	For a collision of two circles with masses $m_i$ and velocities $v_{i-}$ before and $v_{i+}$ after the collision, the impulse exchange  along the line connecting the centers of the circles (the collision normal) is given by these equations\footnote{See Bourg, David M.; Bywalec, Bryan: Physics for Game Developers, O'REILLY, Second Edition, p. 112}:
	
	\begin{align*}
		\Delta p &= m_1\left(v_{1+} - v_{1-}\right),\\
		-\Delta p &= m_2\left(v_{2+} - v_{2-}\right),\\
		e &= -\frac{v_{1+} - v_{2+}}{v_{1-} - v_{2-}}.
	\end{align*}
	
	Note that since $\Delta p$ acts along the collision normal, $v_{1-}$ and $v_{2-}$ in this context are the velocities projected on that normal, given by the projection
	
	\begin{align*}
		\vec{v}_n = \frac{\vec{v}\cdot\vec{n}}{\vec{n}\cdot\vec{n}}\vec{n} \Rightarrow v_n = \vec{v}\cdot\vec{n}.
	\end{align*}

	$e$ is the coefficient of restitution, a value in the interval $[0, 1]$ where $1$ indicates a completely elastic collision 
	 and $0$ indicates an inelastic collision. In atomic collisions, no deformations occur and $e = 1$.\\
	After solving the first equations for $v_{1+}$ and $v_{2+}$, these values are substituted in the last equation:
	
	\begin{align*}
		&v_{1+} = \frac{\Delta p}{m_1} + v_{1-} \hspace{2cm} v_{2+} = -\frac{\Delta p}{m_2} + v_{2-}\\
		&\Rightarrow e = -\frac{\frac{\Delta p}{m_1} + v_{1-} - \left(-\frac{\Delta p}{m_2} + v_{2-}\right)}{v_{1-} - v_{2-}}\\
		&\Leftrightarrow e\left(v_{1-} - v_{2-}\right) = -\left(v_{1-} - v_{2-} + \Delta p\left(\frac{1}{m_1} + \frac{1}{m_2}\right)\right)\\
		&\Leftrightarrow ev_r = -v_r - \Delta p\left(\frac{1}{m_1} + \frac{1}{m_2}\right)\\
		&\Leftrightarrow -\Delta p\left(\frac{1}{m_1} + \frac{1}{m_2}\right) = e v_r + v_r\\
		&\Leftrightarrow \Delta p = -\frac{v_r(e + 1)}{\frac{1}{m_1} + \frac{1}{m_2}}
	\end{align*}

	with the relative veloctiy $v_r = v_{1-} - v_{2-}$. The impulse $\Delta p$ acts along the line of action connecting the center of masses of both circles, so we'll need the normal vector $\vec{n}$ along the collision:
	
	\begin{align*}
		\vec{n} = \frac{\vec{r}_2 - \vec{r}_1}{|\vec{r}_2 - \vec{r}_1|}.
	\end{align*}

	With this, the new velocities are
	
	\begin{align*}
		&\vec{v}_{1+} = \vec{v}_{1-} + \frac{\Delta p}{m_1}\vec{n},\\
		&\vec{v}_{2+} = \vec{v}_{2-} - \frac{\Delta p}{m_2}\vec{n}.
	\end{align*}

	Here, $\vec{v}_{1-}$ and $\vec{v}_{2-}$ are the full velocities before the collision, not just the projection along the normal.

    \subsection{Iterative collision handling}
    We know that for two colliding discs the necessary impulse change along the normal is 

    \begin{equation}
        \label{dp}
        \Delta p = -\frac{v_r(e + 1)}{\frac{1}{m_1} + \frac{1}{m_2}}.
    \end{equation}

    To solve multiple collisions, iterate the collisions and calculate $\Delta p$ using \cref{dp} for each pair, using previously updated velocities directly. Note that $\Delta p \le 0$ if $v_r \ge 0$, meaning the discs are either already separating ($> 0$) or not moving along the collision normal ($= 0$). In that case just let them move apart and/or apply a positional correction based on current overlap. \\

    \cref{dp} can also be used to resolve collisions with boundaries (child or parent membranes) by setting $m_2 = \infty$. \cref{dp} then simplifies to 

    \begin{equation*}
        \Delta p = -m_1v_r(e + 1).
    \end{equation*}

    Just make sure that the normal points in the direction of the impulse change: For disc-child membrane collisions, the normal points from the center of the membrane to the disc, and for disc-parent membrane collisions, the normal points from the disc to the membrane center.

	\section{Reactions}
	
	\subsection{Transformation $A \rightarrow B$}
	
	We just require that $m$ won't change. 
	
	\subsection{Decomposition: $A \rightarrow B + C$}
	
	\subsubsection*{Probability}
	
	The decomposition reaction probability is given in $\frac{\%}{s}$. Let's say the user specified probability $p$ and simulation time step $\Delta t$ in s. The reaction chance is $1 - (1 - p')^N$ and we want a reaction chance of $p$ after $N = \frac{1}{\Delta t}$:
	
	\begin{align*}
		&p = 1 - (1 - p')^{\Delta t^{-1}}\\
		&\Leftrightarrow (1 - p')^{\Delta t^{-1}} = 1 - p\\
		&\Leftrightarrow p' = 1 - \sqrt[\Delta t^{-1}]{1 - p} = 1 - (1 - p)^{\frac{1}{\Delta t^{-1}}}\\
		&\Rightarrow p' = 1 - (1 - p)^{\Delta t}
	\end{align*}
	
	\subsubsection*{Physics}
	
	Let A have $\vec{p} = (m_1 + m_2)\vec{v}$. B and C should move in opposite directions perpendicular to $\vec{v}$ after the decomposition, conserving energy and momentum. We only need conservation of momentum to see:
	
	\begin{align*}
		(m_1 + m_2)v = m_1v_1 + m_2v_2 \Rightarrow v_1 = v_2 = v
	\end{align*}

	Since we want B and C to move away perpendicular from the previous direction, we'll just multiply $v$ with 2 perpendicular unit vectors:
	
	\begin{align*}
		\vec{n} = \frac{\vec{v}}{v} \hspace{20px} \vec{v}_1 = v\begin{pmatrix} -n_y\\ n_x\end{pmatrix} \hspace{20px} \vec{v}_2 = v\begin{pmatrix} n_y\\ -n_x\end{pmatrix}
	\end{align*}
	
	\subsection{Combination: $A + B \rightarrow C$}
	This is just a classical inelastic collision (see wikipedia for derivation):
	
	\begin{align*}
		\vec{v} = \frac{m_1 \vec{v}_1 + m_2 \vec{v}_2}{m_1 + m_2}
	\end{align*}

	Note that after this collision, $C$ gained internal energy:
	
	\begin{align*}
		\Delta U_C = \frac{1}{2}\frac{m_1 m_2}{m_1 + m_2}(v_1 - v_2)^2
	\end{align*}

	This internal energy will lead to the reverse reaction $C \rightarrow A + B$ if $U_C > E_a$, where $E_a$ is the activation energy required for the forward reaction.
	
	\subsection{Exchange: $A + B \rightarrow C + D$}
	This is handled like a normal collision, we just change the types after. We handle this in terms of 2 separate transformations $A \rightarrow C$ and $B \rightarrow D$, conserving energy:
	
	\begin{align*}
		m_A v_A^2 = m_C v_C^2 \Leftrightarrow v_C = \sqrt{\frac{m_A}{m_C}}v_A
	\end{align*}

	and mass
	
	\begin{align*}
		m_A + m_B = m_C + m_D
	\end{align*}
	
	This does not conserve momentum and is wrong until internal and activation energies are taken into account (TODO).

\end{document}