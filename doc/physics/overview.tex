\documentclass{article}

\usepackage{amsmath}
\usepackage{listings}
\usepackage{xcolor}
\usepackage{cleveref}

\lstset{
	language=Python,
	backgroundcolor=\color{white},
	basicstyle=\ttfamily,
	keywordstyle=\color{blue},
	stringstyle=\color{red},
	commentstyle=\color{green},
	numberstyle=\tiny\color{gray},
	numbers=left,
	stepnumber=1,
	numbersep=10pt,
	showstringspaces=false,
	tabsize=4,
}

\begin{document}
	\section{Collision physics}
	\subsection{Collision with the bounds}
	\subsubsection{Rectangular bounds}
	We'll describe the rectangular bounds by their 2 corner points $(x_{min}, y_{min})$ and $(x_{max}, y_{max})$. 
	The circle is at position $(x(t), y(t))$ with radius $R$. The penetration $l$ through the wall is $l > 0$ if the circle is overlapping with the bounds. To handle the collision, negate the offending velocity component and move the disc $\Delta x_i = \pm 2 l$ along the respective axis:

	\subsubsection*{Left wall}
	\begin{align*}
		l(t) &= R + x_{min} - x(t)\\
		\Delta x &= 2 l\\
		v_x' &= -v_x
	\end{align*}

	\subsubsection*{Right wall}
	\begin{align*}
		l(t) &= R - x_{max} + x(t)\\
		\Delta x &= - 2 l\\
		v_x' &= -v_x
	\end{align*}

	\subsubsection*{Top wall}
	\begin{align*}
		l(t) &= R + y_{min} - y(t)\\
		\Delta y &= 2 l\\
		v_y' &= -v_y
	\end{align*}

	\subsubsection*{Bottom wall}
	\begin{align*}
		l(t) &= R - y_{max} + y(t)\\
		\Delta y &= - 2 l\\
		v_y' &= -v_y
	\end{align*}
	
	\subsection{Overlap between circles}
	The overlap $l$ of 2 circles is the sum of their radii minus the distance between them. If the overlap is $> 0$, we have a collision. The current positions of the circles are $\vec{r}_{i, 0}$.
	\begin{equation}
		l = R_1 + R_2 - \left|\vec{v}_{2, 0} - \vec{v}_{1, 0}\right|
	\end{equation}
	We want to move the circles to the positions they had just before the collision (when they were just touching without intersection). For this, we need to move them back in time using their velocities multiplied with a time $\Delta t$ we have to calculate. At that time, the overlap should be 0:
	
	\begin{align*}
		l(t) &= R_1 + R_2 - \left|\vec{r_2}(t) - \vec{r_1}(t)\right|\\
	 &= R_1 + R_2 - \left|\vec{r}_{2, 0} + t\cdot\vec{v_2} - \vec{r}_{1, 0} - t\cdot\vec{v_1}\right|\\
	 &= R_1 + R_2 - \left|\vec{r}_{2, 0} - \vec{r}_{1, 0} + t\left(\vec{v_2} - \vec{v_1}\right)\right|\\
	 &= R_1 + R_2 - \left|\vec{r} + t\cdot\vec{v}\right| \overset{!}{=} 0\\
	 &\Leftrightarrow \sqrt{\left(\left(r_x+t\cdot v_x\right)^2+\left(r_x+t\cdot v_y\right)^2\right)} = R_1 + R_2\\
	 &\Rightarrow \Delta t = \frac{-r_x v_x - r_y v_y \pm \sqrt{-r_x^2 v_y^2 + 2 r_x r_y v_x v_y - r_y^2 v_x^2 + (R_1 + R_2)^2 (v_x^2 + v_y^2)}}{v_x^2 + v_y^2}
	\end{align*}

	with distance vector $\vec{r} = \vec{r_{2, 0}} - \vec{r_{1, 0}}$ and relative velocity $\vec{v} = \vec{v_2} - \vec{v_1}$ of the circles. The 2 solutions for $\Delta t$ refer to the first point of contact before overlap and the point in time exactly after the circles have passed through each other. The first contact happens at an earlier time, so we'll select the earlier of the 2 solutions, the one with $-\sqrt{...}$.\\
	Now the correct initial positions $\vec{r}_{i, c}$ at the time of collision can easily be calculated:
	\begin{equation}
		\vec{r}_{i, c} =\vec{r}_{i, 0} + \Delta t \cdot \vec{v}_i
	\end{equation}
	After the collision was handled and the new velocities $\vec{v}'_i$ have been calculated, the circles need to be wound forward in time with their corrected velocities:
	\begin{equation}
		\vec{r}_{i} =\vec{r}_{i, c} - \Delta t \cdot \vec{v}'_i
	\end{equation}
	
	\begin{lstlisting}[caption={Code to calculate $t$}]
from sympy import symbols, Eq, solve, simplify, sqrt

rx, ry, vx, vy, R1, R2, t = symbols("rx ry vx vy R1 R2 t")
eq = Eq(sqrt((rx + t*vx)**2+(ry+t*vy)**2), R1+R2)
solution = solve(eq, t)

simplified = [simplify(sol) for sol in solution]

for sol in simplified:
    print(sol)
	\end{lstlisting}

	\subsection{Collision handling}
		
	For a collision of 2 circles with masses $m_i$ and velocities $v_{i-}$ before and $v_{i+}$ after the collision, the impulse exchange  along the line connecting the centers of the circles (the collision normal) is given by these equations\footnote{See Bourg, David M.; Bywalec, Bryan: Physics for Game Developers, O'REILLY, Second Edition, p. 112}:
	
	\begin{align*}
		\Delta p &= m_1\left(v_{1+} - v_{1-}\right),\\
		-\Delta p &= m_2\left(v_{2+} - v_{2-}\right),\\
		e &= -\frac{v_{1+} - v_{2+}}{v_{1-} - v_{2-}}.
	\end{align*}
	
	Note that since $\Delta p$ acts along the collision normal, $v_{1-}$ and $v_{2-}$ in this context are the velocities projected on that normal, given by the projection
	
	\begin{align*}
		\vec{v}_n = \frac{\vec{v}\cdot\vec{n}}{\vec{n}\cdot\vec{n}}\vec{n} \Rightarrow v_n = \vec{v}\cdot\vec{n}.
	\end{align*}

	$e$ is the coefficient of restitution, a value in the interval $[0, 1]$ where $1$ indicates a completely elastic collision 
	 and $0$ indicates an inelastic collision. In atomic collisions, no deformations occur and $e = 1$.\\
	After solving the first equations for $v_{1+}$ and $v_{2+}$, these values are substituted in the last equation:
	
	\begin{align*}
		&v_{1+} = \frac{\Delta p}{m_1} + v_{1-} \hspace{2cm} v_{2+} = -\frac{\Delta p}{m_2} + v_{2-}\\
		&\Rightarrow e = -\frac{\frac{\Delta p}{m_1} + v_{1-} - \left(-\frac{\Delta p}{m_2} + v_{2-}\right)}{v_{1-} - v_{2-}}\\
		&\Leftrightarrow e\left(v_{1-} - v_{2-}\right) = -\left(v_{1-} - v_{2-} + \Delta p\left(\frac{1}{m_1} + \frac{1}{m_2}\right)\right)\\
		&\Leftrightarrow ev_r = -v_r - \Delta p\left(\frac{1}{m_1} + \frac{1}{m_2}\right)\\
		&\Leftrightarrow -\Delta p\left(\frac{1}{m_1} + \frac{1}{m_2}\right) = e v_r + v_r\\
		&\Leftrightarrow \Delta p = -\frac{v_r(e + 1)}{\frac{1}{m_1} + \frac{1}{m_2}}
	\end{align*}

	with the relative veloctiy $v_r = v_{1-} - v_{2-}$. The impulse $\Delta p$ acts along the line of action connecting the center of masses of both circles, so we'll need the normal vector $\vec{n}$ along the collision:
	
	\begin{align*}
		\vec{n} = \frac{\vec{r}_2 - \vec{r}_1}{|\vec{r}_2 - \vec{r}_1|}.
	\end{align*}

	With this, the new velocities are
	
	\begin{align*}
		&\vec{v}_{1+} = \vec{v}_{1-} + \frac{\Delta p}{m_1}\vec{n},\\
		&\vec{v}_{2+} = \vec{v}_{2-} - \frac{\Delta p}{m_2}\vec{n}.
	\end{align*}

	Here, $\vec{v}_{1-}$ and $\vec{v}_{2-}$ are the full velocities before the collision, not just the projection along the normal.

	\section{Reactions}

    \subsection{Overview}
    Every disc has internal energy $U$ and kinetic energy $E_k$. Every disc type has a user-defined enthalpy of formation $H_f$. This results in all reactions having a reaction enthalphy $H = H_f(\text{products}) - H_f(\text{educts})$. Every collision has a fixed probability $p$ to exchange a random small fraction $f$ of kinetic and inner energy between each other. All user-defined reactions must satisfy conservation of mass.\\
    An overview of the 4 reaction types is given in \cref{tab_reactions}.

    \begin{table}[h!]
        \centering
        \begin{tabular}{|l|l|l|}
            \hline
            \textbf{Reaction Type} & \textbf{Requirements} & \textbf{Energy Changes} \\
            \hline
            \begin{tabular}[c]{@{}l@{}}Transformation:\\ A $\rightarrow$ B\end{tabular} & $U(A) \geq H$ & 
            \begin{tabular}[c]{@{}l@{}}Reduce $U(A)$ by $H$;\\ then turn A into B\end{tabular} \\
            \hline
            \begin{tabular}[c]{@{}l@{}}Decomposition:\\A $\rightarrow$ B + C\end{tabular} & $U(A) \geq H$ & 
            \begin{tabular}[c]{@{}l@{}}Reduce $U(A)$ by $H$;\\ divide remaining $U$ and $E_k$\\ proportionally to masses of B and C\end{tabular} \\
            \hline
            \begin{tabular}[c]{@{}l@{}}Combination:\\A + B $\rightarrow$ C\end{tabular} & $U(A) + U(B) + E_k(A) + E_k(B) \geq H$ & 
            \begin{tabular}[c]{@{}l@{}}Create C by inelastic collision;\\ turn lost $E_k$ into $U(C)$;\\ reduce random portions of $U(C)$\\ and $E_k(C)$ by $H$ total\end{tabular} \\
            \hline
            \begin{tabular}[c]{@{}l@{}}Exchange:\\A + B $\rightarrow$ C + D\end{tabular} & 
            \begin{tabular}[c]{@{}l@{}}\\ $U(A) + U(B) + E_k(A) + E_k(B) \geq H$\end{tabular} & 
            \begin{tabular}[c]{@{}l@{}}Reduce $U$ and $E_k$ of educts by $H$\\ (random portions for each);\\ divide remaining $U$ and $E_k$\\ proportionally to masses of C and D\\ Choose products closest in mass\end{tabular} \\
            \hline
        \end{tabular}
        \caption{Reaction types, requirements, and energy changes.}
        \label{tab_reactions}
    \end{table}

    Each forward reaction automatically creates a backward reaction with $H_b = -H_f$. Reaction probabilities are still required to be calculated (see below). The system checks if there are any unimolecular reactions without defined pre-exponential factors.

	\subsection{Transformation $A \rightarrow B$}
	
	No additional thoughts.
	
	\subsection{Decomposition: $A \rightarrow B + C$}
    We don't conserve momentum here, maybe fix later.\\
	B and C should move in opposite directions perpendicular to $\vec{v}$ after the decomposition:
	
	\begin{align*}
		\vec{v}_1 = \begin{pmatrix} -v_y\\ v_x\end{pmatrix} \hspace{20px} \vec{v}_2 = \begin{pmatrix} v_y\\ -v_x\end{pmatrix}
	\end{align*}
	
	\subsection{Combination: $A + B \rightarrow C$}
	This is just a classical inelastic collision:
	
	\begin{align*}
		\vec{v} = \frac{m_1 \vec{v}_1 + m_2 \vec{v}_2}{m_1 + m_2}
	\end{align*}

	Note that after this collision, $C$ gained internal energy (see wikipedia for derivation):
	
	\begin{align*}
		\Delta U_C = \frac{1}{2}\frac{m_1 m_2}{m_1 + m_2}(v_1 - v_2)^2
	\end{align*}
	
	\subsection{Exchange: $A + B \rightarrow C + D$}
	\Cref{tab_reactions} states to choose products closest in mass. Meaning: If $A + B \rightarrow C + D$ with $m(A) = 100$, $m(B) = 5$, $m(C) = 101$ and $m(D) = 4$, turn $A \rightarrow C$ and $B \rightarrow D$ because they are closer in mass.

    \subsubsection{Probability}
    \subsubsection*{Unimolecular reactions}
    We can use Arrhenius law for the rate constant of reactions:

    \begin{equation*}
        k = A\exp{\left(-\frac{E_a}{k_BT}\right)},
    \end{equation*}

    where $k$ is the rate constant, $A$ is the attempt frequency (pre-exponential factor) and depends on the disc type, $E_a$ is the activation energy and $k_BT$ is the average energy of the particles. Note that $A$ is user-defined and translates to reactions per second (if enough energy is available). Also, for exothermic reactions the e-function should be replaced by 1. In this simulation, the activation energy is the reaction enthalpy $H$ and the energy of the particles relevant for reactions is their internal energy $U$. The rate constant can already be viewed as the frequency of collisions resulting in a reaction which is precisely the reaction probability $p$:

    \begin{equation*}
        p = A\exp{\left(-\frac{H}{U}\right)}.
    \end{equation*}

	Now we need to calculate the probability based on the simulation time step $\Delta t$ in s. The reaction chance is $1 - (1 - p')^N$ and we want it to be equal to $p$ after $N = \frac{1}{\Delta t}$:
	
	\begin{align*}
		&p = 1 - (1 - p')^{\Delta t^{-1}}\\
		&\Leftrightarrow (1 - p')^{\Delta t^{-1}} = 1 - p\\
		&\Leftrightarrow p' = 1 - \sqrt[\Delta t^{-1}]{1 - p} = 1 - (1 - p)^{\frac{1}{\Delta t^{-1}}}\\
		&\Rightarrow p' = 1 - (1 - p)^{\Delta t}
	\end{align*}

    \subsubsection*{Bimolecular reactions}
    In a realistic model, reactions can only occur if the reacting interface of the colliding particles are actually close to another. This can be modelled by defining how much of a disc's circumference is "reactive". With the circumference $C_A = 2\pi r_A$ and the reactive arc length $s_A$, the probability of a reaction for colliding discs A and B is given by 

    \begin{equation*}
        p = \frac{s_A}{C_A}\cdot\frac{s_B}{C_B}.
    \end{equation*}
\end{document}