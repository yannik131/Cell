\documentclass{article}

\usepackage{caption}
\usepackage{graphicx}
\usepackage{float}
\usepackage{enumitem}
\usepackage{hyperref}

\begin{document}
	\section{Cell cycle}
	
	\begin{figure}[H]
		\centering
		\includegraphics[width=0.8\linewidth]{durations.jpg}
		\caption{\url{https://dehistology.blogspot.com/2011/06/cell-cycle.html}}
	\end{figure}
	
	\subsection{Cdks}
	Cyclin-Dependent Kinases (CDKs) phosphorylate proteins that drive the cell through the cell cycle. 
	
	\subsection{Cyclins}
	Cyclins activate the CDKs (partially) by forming complexes. See table.
	
	\begin{table}[h]
		\centering
		\begin{tabular}{|c|c|c|} \hline
			Cyclin & Cdks & Complex \\ \hline
			Cyclin-D & (Cdk4/Cdk6)  & G1-Cdk \\ \hline
			Cyclin-E & Cdk2         & G1/S-Cdk \\ \hline
			Cyclin-A & (Cdk2/Cdk1)  & S-Cdk (MPF) \\ \hline
			Cyclin-B & Cdk1         & M-Cdk (MPF)\\ \hline
		\end{tabular}
		\caption{Cyclines and the Cdks they form complexes with}
	\end{table}

	\begin{enumerate}[label=\textbullet]
		\item G1/S-Cdk initiates the cell cycle in the dormant G1 phase  (yeast: START, mammals: restriction point)
		\item S-Cdk triggers transition from the G1 to the S phase 
		\item M-Cdk triggers transition from the G2 to the M phase 
	\end{enumerate}

	The resulting complex when Cyclin A or B bind to Cdk1/Cdk2 is called the maturation promoting factor (MPF).
	
	\begin{figure}[H]
		\centering
		\includegraphics[width=\linewidth]{mpf_regulation_cooper.png}
		\caption{MPF Regulation}
	\end{figure}

	Cdk concentration remains mostly constant.\footnote{albert, p. 1093} Concentration of cyclins changes during the cell cycle (see images).
	
	\begin{figure}[H]
		\centering
		\includegraphics[width=\linewidth]{cyclin_activity_wikipedia.png}
		\caption{Concentration of the cyclins during the cell cycle (wikipedia)}
	\end{figure}
	
	\begin{figure}[H]
		\centering
		\includegraphics[width=\linewidth]{cyclin_activity_alberts.png}
		\caption{Concentration of the cyclins during the cell cycle (alberts)}
	\end{figure}
	
	\begin{figure}[H]
		\centering
		\includegraphics[width=0.8\linewidth]{cyclin_activity_karp.png}
		\caption{Concentration of the cyclins during the cell cycle (karp). cdc2 = cdk1}
	\end{figure}
	
	\subsection{Wee1}
	Wee1 inhibits Cyclin-Cdk complexes by phosphorylation, causing a delay of the M-Phase so that the cell can grow. If Wee1 is defective, the cells transition directly from S- to M-Phase without the growth in the G2 phase, resulting in \textit{wee} little cells :)
	
	\subsection{Cdc25}
	Cdc25 activates the MPF that was previously inactivated by Wee1. It dephosphorylizes Thr14 and Tyr15. Mammals have 3 related forms Cdc25A, B and C.
	
	\subsection{APC/C}
	The anaphase-promoting complex/cyclosome (APC/C) is a ubiquitin ligase and triggers transition from the metaphase to the anaphase by ubiquitylation of securin and Cyclin-A/Cyclin-B, marking them for proteolysis. Once they are destroyed, the MPF is no more.
	
	\subsubsection{Cdc20}
	Binds with APC/C in mitosis to specify target proteins.
	
	\subsubsection{Cdh1}
	Binds with APC/C in late mitosis/G1 to specify target proteins.
	
	\subsection{SCF}
	Another ubiquitin ligase. CF ubiquitylates CKIs in the late G1 phase, thus activating S-Cdks; it's also responsible for proteolysis of G1/S-Cdks in the early S phase. Marks p27. Typically requires phosphorylated targets.
	
	\subsubsection{F-Box-Proteins}
	Exchangeable part of the SCF, specifying the target protein. There are more than 70 genes coding for F-Box-Proteins.\footnote{alberts, p. 178}
	
	\subsection{CKIs}
	Cdk inhibitors (CKIs) inhibit Cyclin-Cdk complexes by binding to them (mostly G1/S- and S-Cdks). There are 2 families of CKIs, binding to different Cdks and Cyclin/Cdk complexes:
	
	\begin{figure}[H]
		\centering
		\includegraphics[width=\linewidth]{ckis_cooper.png}
		\caption{CKI families and their targets}
	\end{figure}

	\subsubsection{p27}
	Inhibits Cdks in G1. Gets phosphorylated by Cdk1, causing it to be marked for proteolysis (by SCF or APC/C?). 
	
	\subsubsection{p21}
	Inhibits G1/S-Cdk und S-Cdk if DNA damage occured.
	
	\subsubsection{p16}
	Inhibits G1-Cdk in G1. Frequently inactive in cancer cells.

\end{document}